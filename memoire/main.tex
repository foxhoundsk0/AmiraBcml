\documentclass[a4paper,12pt]{report}
\usepackage[francais]{babel}
\usepackage[utf8]{inputenc}
\usepackage{graphicx}
\graphicspath{ {images/}}
\usepackage[width=150mm,top=25mm,bottom=25mm,bindingoffset=6mm]{geometry} 
\usepackage[hidelinks]{hyperref} %Pour pouvoir cliquer sur la table des matières...
\usepackage{fancyhdr} 
\pagestyle{fancy}
\renewcommand{\headrulewidth}{0.4pt} %pour  tracer une ligne au dessous l'entete
\renewcommand{\footrulewidth}{0.4pt} %pour  tracer une ligne au dessus le pied de la page
\usepackage[style=numeric, backend=biber]{biblatex}
\addbibresource{references.bib}

\begin{document}
\begin{titlepage}
    \begin{center}
        \vspace*{1cm}
        \Huge
        \vspace{0.5cm}
        \LARGE
        \textbf{LaTeX: Écrire son mémoire pas par pas}\\
        \vspace{1.5cm}
        \Large
		Amira Bouchama
        \vfill
        \large
		Programme Mobadara\\
 		 \vspace{0.8cm}
 		 \includegraphics[width=0.4\textwidth]{MobadaraProgram}\\
         \small
        02 Juillet 2020
    \end{center}
\end{titlepage}
\newpage
\chapter*{Remerciments}
\begin{center}
Je vous remercie d'avoir assister à la formation.\\ % saut de la ligne
\large{\textbf{Bon courage :)}} % ecrit en large et gras
\end{center} %le texte est mis en centre
 

\tableofcontents
\listoffigures
\listoftables
\chapter*{Introduction}
\addcontentsline{toc}{chapter}{Introduction générale}
Vous mettez ici l'introduction de votre projet.
\chapter{Premier chapitre}
\section{Première section}
cette section contient deux sous sections:
\subsection{Sous section 1}
$$E(X)=\frac{a}{b}$$ % formule math centrée
\begin{equation}
Var(X)=\frac{a}{b^{2}}
\end{equation} % equation numérotée
\\Voici une note\footnotemark \footnotetext{La définition de la note}
\newpage
\subsection{Sous section 2}
cette sous section contient deux sous sous sections:
\subsubsection{Sous sous section 1}
Une liste:
\begin{itemize}
\item[\textbullet] premier item
\item[*] deuxième item
\end{itemize}
\subsubsection{Sous sous section 2}
La figure \ref{info pour se referer}
\begin{figure}[h]
\centering
\includegraphics[scale=1]{MobadaraProgram.png}
\caption{Mobara Programme}  
\label{info pour se referer} 
\end{figure} 
\\cette sous sous section contient un paragraphe:
\paragraph{paragraphe}
Hello ! 
Le tableau \ref{abc}
\begin{table}[h]
\centering
\begin{tabular}{| l | l | l |}
\hline
L1C1 & L1C2 & L1C3 \\
\hline
L2C1 & L2C2 & L2C3 \\
\hline
L3C1 & L3C2 & L3C3 \\
\hline
\end{tabular}
\caption{caption du tableau}
\label{abc}
\end{table}

Pour la documentation consulter ce lien \cite{overleaf}


\chapter*{Conclusion}
\addcontentsline{toc}{chapter}{Conclusion générale}
Ici vous mettez la conclusion du projet.
\appendix
\chapter{Nom Annexe}
\input{chapitres/annexe}
\printbibliography
\chapter*{Résumé}
Ici vous mettez le résumé.
\end{document}

